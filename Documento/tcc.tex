\documentclass[a4paper, 12pt]{article}
%pacote para definir as margens
\usepackage[top=2cm, bottom=2cm, left=2.5cm, right=2.5cm]{geometry}
%pacote para colocar acentos
\usepackage[utf8]{inputenc}
%pacote para incluir comentários em multiplas linhas
\usepackage{verbatim}
\begin{document}
% capa
\begin{titlepage} %iniciando a "capa"
\begin{center} %centralizar o texto abaixo
{\large PONTIFÍCIA UNIVERSIDADE CATÓLICA DE MINAS GERAIS}\\[0.2cm] %0,2cm é a distância entre o texto dessa linha e o texto da próxima
{\large PUC Minas Virtual}\\[0.2cm] % o comando \\ "manda" o texto ir para próxima linha
{\large Ciência de Dados e Big Data}\\[5.1cm]
{\textbf PySnake - Ciência de Dados e Big Data para Penteste}\\[5.1cm] % o comando \bf deixa o texto entre chaves em negrito. O comando \huge deixa o texto enorme
\end{center} %término do comando centralizar
{\large Aluno: Marcelo Pontes Rodrigues}\\[0.7cm] % o comando \large deixa o texto grande
{\large Professor:}\\[5.1cm]
\begin{center}
{\large Brasília - DF}\\[0.2cm]
{\large 2020}
\end{center}
\end{titlepage} %término da "capa"
\pagebreak[1]

\section{INTRODUÇÃO}
	\subsection{O tratamento dos dados}
		\paragraph{...}

\section{MATERIAIS E MÉTODOS}
	\subsection{Equipametnos, softwares/ferramentas e dados utilizados no projeto}
	\subsection{Metodologia de desenvolvimento}

\section{RESULTADOS}
	\subsection{Visão geral do sistema proposto}
	\subsection{Especificação do Sistema}
	\subsubsection{Modelagem do processo de Pentest}
	\subsection{Implementação do sistema}
	\subsection{Considerações finais}

\section{DISCUSSÃO}

\section{CONCLUSÃO}

\section{REFERÊNCIAS}

\end{document}
